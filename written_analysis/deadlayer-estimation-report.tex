\documentclass[11pt]{article}

\usepackage{graphicx}
\usepackage{amsmath}
\usepackage{amsfonts}
\usepackage[margin = 1.00 in]{geometry}
\usepackage{relsize}
\usepackage[normalem]{ulem}
%\usepackage{fontspec}
%\setmainfont{Calibri}


\usepackage{enumitem}



% input the problem number 
\newenvironment{problem}[1]
{
	\flushleft
	\textbf{Problem #1.} 
	\itshape
}


\usepackage{amssymb}
\newcommand*{\QEDA}{\hfill\ensuremath{\blacksquare}}%
\newenvironment{solution}
{
	\flushleft
	\textit{Solution}. 
}
{
	\QEDA
}

\usepackage{mathtools}
\DeclarePairedDelimiter{\ceil}{\lceil}{\rceil}

\newcommand{\eqrefn}[1]{Equation \ref{eq:#1}}


\newcommand\numberthis{\addtocounter{equation}{1}\tag{\theequation}}



\newcommand{\dE}{\; dE } 
\newcommand{\dx}{\, dx}
\newcommand{\dy}{\, dy}
\newcommand{\du}{\, du}
\newcommand{\dt}{\, dt}
\newcommand{\dtheta}{\, d\theta}
\newcommand{\dz}{\, dz}
\newcommand{\sgn}{\text{sgn} \,}

\usepackage{indentfirst}

\usepackage{hyperref}

\usepackage{footnote}
\usepackage{mhchem}
\usepackage{hyperref}

\usepackage{subcaption}
\DeclareCaptionSubType*{figure}
\captionsetup[subfigure]{labelformat=simple, labelsep=colon}

\usepackage[outdir=./]{epstopdf}
\usepackage{epstopdf}

\usepackage{siunitx}
\sisetup{output-exponent-marker=\ensuremath{\mathrm{e}}}

\usepackage {txfonts}

\usepackage{multicol}
\usepackage{hyperref}



%%%%%%%%%%




\title{Simultaneous Estimation of Normalized Source Dead-Layer and Detector Dead-Layer in a Double-Sided Strip Detector \\ (Draft 1)}
\author{Jacob Pierce \\ \href{mailto:jacobpierce@uchicago.edu}{jacobpierce@uchicago.edu} }
\date{12.04.17}

\begin{document}
\thispagestyle{empty}

\maketitle


\begin{abstract}


\end{abstract}




\begin{multicols}{2}


\section{Introduction} 





\section{Procedure}

Here we describe the technique used to analyze the data. A detailed procedure of how the data was collected can be found elsewhere. 


\subsection{The Model}

In this section, we will progress through an increasingly complicated set on scenarios which lead to the model upon which we will regress to determine the detector dead-layer depth.

Suppose a charged particle is emitted by a decaying radioactive source and subsequently enters an ionizing radiation detector where the charge from scattered particles is measured, and finally histogrammed as a channel via digitizer. We assume that there exists a linear relationship between the energy of the particle as it enters the part of the detector responsible for ionization of charge.%
\footnote{In general, a polynomial relationship could also be used with equal difficulty in the implementation discussed later.}
%
In practice, energy loss occurs before the charged particle enters the part of the detector in which charge is actually measured. Two main effects contribute to this loss: energy loss as the particle travels through an inactive part of the detector, known as the \textbf{detector dead-layer}, and as the particle travels through the source it is emitted from. In particular, if the source and detector are flat and both have small dimensions compared to their distance of separation, then all charged particles leave the source at a constant angle $\phi$ and enter the detector another constant angle $\theta$. If the source and detector respectively have orthogonal distances $ d_S $ and $ d_D $ and stopping power functions $S_S(E)$ and $S_S(D)$, then we expect the relationship
%
	\begin{align*}
 		E_\text{	measured} = E_0 &- \int_{0}^{ d_S \sec (\phi) } S_S(E_S(x)) \dx 
			\\ &- \int_{0}^{ d \sec( \theta)} S_D(E_D(x)) \dx
				\numberthis 	\label{eq:e_measured}
	\end{align*}
where in both cases, $E(x)$ is determined implicitly as the inverse of invertible function $x(E) = \int_{E_0}^E dE / S(E) $. Note that such an inverse exists because $x(E)$ increases monotonically with $E$ since $S(E) > 0$.

If the distances $d_S$ and $d_D$ are both small, then \eqrefn{e_measured} can be simplified considerably by assuming the $S$ is constant over the region of integration, yielding
	\begin{align*}
 		E_\text{measured} = E_0 &- d_S  S_S(E_0)  \sec (\phi) 
			\\ &-  d_D S_D( E_0 - u\sec (\phi ) ) \sec \theta
				  \numberthis \label{eq:s_linear}
	\end{align*}	
Hence the energy measured is linear in $\sec(\theta)$ and approximately linear in $ \sec(\phi) $. %For an alpha particle, which is what we study in this report, the energy of the measured particle 

Now in a two-sided strip-detector, each strip is held at the same voltage and shares the same relationship between channel (e.g. of the peak in the spectrum or the $\mu$ parameters of the Bortels function, to be discussed shortly) and energy. Moreover, because the detector has two-sides with strips running orthogonal, we can experimentally determine which pixel an incident particle hit. We can independently determine the  the angles $\theta_{ij}$ and $\phi_{ij}$ given adequate information about the relative geometry of the detector and a source. \eqrefn{s_linear} holds separately for each pixel, so if we observe the extract the position of each peak $p_{ij}$, we expect that there exist $A_i$ and $B_i$ for each strip such that we have the relationship:
	\begin{align*}
 		A_i p_{ij} + B_i = E_0  &-  d_D  S_D( E_0 - m \sec (\phi _{ij}) ) \sec \theta_{ij}
			\\ &- m  \sec (\phi_{ij}) ,
				  \numberthis \label{eq:1peak}
	\end{align*}	
where we have relabeled the constant $ d_S S_{S(E_0)} \equiv m $. We call $m$ the \textbf{source dead-layer} and $d_D$ the \textbf{detector dead-layer}; note that the detector dead-layer has units of distance while the source dead-layer has units of energy.

Now suppose that multiple alpha peaks are observed in the spectrum, known to originate with energy $E_{0k}$. Then \eqrefn{1peak} holds separately for each peak with a different $m\to m_k$ for each peak, since the stopping power of one or multiple sources varies with energy. However, the linear calibration of each strip is independent of $k$, as are the coordinates of each pixel and hence $\theta_{ij}$ and $\phi_{ij}$ as well. Thus, we expect the relationship:
	\begin{align*}
 		A_i p_{ijk} + B_i = E_{0k}  &-  d_D  S_D( E_{0k} - m_k \sec (\phi ) ) \sec \theta_{ij}
			\\ &- m_k \sec (\phi_{ij}) 
				  \numberthis \label{eq:fitfunction_1dataset}
	\end{align*}	
	
We now add the final complication of multiple datasets, labeled by $l$. Our data was obtained by setting the detector voltage to 0, adjusting sources, and then turning the detector voltage back on. When the sources are moved, we will obtain new angles $\theta_{ijl}$ and $\phi_{ijl}$, still independent of the peak index. We cannot assume that the linear calibration of each strip remains the same after resetting the detector, so the linear calibration coefficients are indexed as $ A_{il} $ and $B_{il}$. The peak positions shift between datasets, so they are indexed as $ p_{ijkl}$. However, $m_k$ and $E_{0k}$ are properties of the sources which are unchanged between different datasets. Applying \eqrefn{fitfunction_1dataset} to each dataset and isolating $E_0{k}$, we obtain the model
	\begin{align*}
 		 E_{0k}  &= A_{il} p_{ijkl} + B_{il } +   + m_k \sec (\phi_{ijl}) 
			\\ &  + d_D  S_D( E_{0k} - m_k \sec (\phi_{ijl} ) ) \sec \theta_{ijl}
				  \numberthis \label{eq:fitfunction}
			\\ & \equiv M(A_{il}, B_{il}, d_D, m_k ; p_{ijkl}, \phi_{ijl} ,  \theta_{ijl} )
	\end{align*}	
where $(A_{il}, B_{il}, d_D, m_k )$ are parameters and   $ (p_{ijkl}, \phi_{ijl} ,  \theta_{ijl}) $ is one data point, both defined for each $ijkl$. 
If our data is consistent with such a model, then minimizing the quantity 
\[  \chi^2  = \sum_{ ijkl} \frac{ ( E_{0k} - M)^2 }{ \Delta ( E_{0k} -  M )^2 } 
	\numberthis \label{eq:chi2}\]
with respect to the parameters should converge with $ \tilde{\chi}^2 \equiv \chi^2 / (N - D) \sim 1$, where $N$ is the total number of data-tuples $ (p_{ijkl}, \phi_{ijl} ,  \theta_{ijl} ) $ and $D$ is the number of degrees of freedom in the fit. In addition, the detector manufacturer has given an estimate $d_D = 100 \pm 25 $ nm, so we expect a convergent regression to yield a $d_D$ in this vicinity.





\subsection{The Dataset}

We obtained four independent datasets between whose collection the voltage of the detector was reset. Hence, $l \in 1 \hdots 4$. The strip detectors used have 32 strips running orthogonally on either side, so $i,j \in 1 \hdots 32 $. Finally, in each dataset, the detector recorded spectra of three distinct sources, of which two peaks were used. This gives a total of $2 \cdot 3 = 6 $ peaks, so $ k \in 1 \hdots 6$. All of the sources were parallel to the detector except in one dataset, in which one of the sources was rotated relative to the plane of the detector; this particular source allows decoupling of the source and detector dead-layers.

It is worth noting that the regression in \eqrefn{chi2} could, in principle, be performed if there were only one independent dataset ($l$ fixed) for spectra with only one peak ($k=1$) as long as the source contributing the one peak was rotated relative to the detector so that $\theta_{ij} \neq \phi_{ij} $. In this sense, our regression is highly constrained, having far more data than is reasonably necessary to achieve a good fit.





\section{Analysis}



\subsection{Fitting} 



\subsection{Peak Extraction}



\subsection{Computation of $\theta_{ijl}$, $\phi_{ijl}$}



\subsection{Simultaneous Calibration}

Now that the data $ ( p_{ijkl}, \theta_{ijl}$, $\phi_{ijl} ) $ have been obtained for all $i,j,k,l$, we may proceed to minimize \eqrefn{chi2} as a function of the parameters. In order to approximate the stopping power of the detector, we took a 

We once again used the \texttt{lmfit} Python library with the Levenberg-Marquardt least-squares algorithm.\cite{lmfit} When all four available datasets were combined, we obtained a good fit with


\section{Conclusion}






\section{Appendix}

\subsection{Alternate Models}


 

%\begin{figure}
%\centering
%  \includegraphics[width=.8\linewidth]{}
%  \caption{Electronics for the timing experiment. A pulse from the PMT is appropriately discriminated. The output is sent to the TAC stop input, and a delayed signal is sent to the TAC start input. A second pulse from the PMT within the TAC range will stop the TAC and the TAC output signal is binned on a PHA.}
%  \label{fig:electronics}
%\end{figure}

%   \begin{figure}[h]
%  \begin{center}
%\begin{tabular}{c c}
%\begin{subfigure}{.47\linewidth}\centering\fbox{\includegraphics[width=\linewidth]{../images/calibration_day1}}\caption{Calibration 1.}\label{fig:taba}\end{subfigure} & 
%\begin{subfigure}{.47\linewidth}\centering\fbox{\includegraphics[width=\linewidth]{../images/calibration_day2}}\caption{Calibration 1.}\label{fig:taba}\end{subfigure}
%\end{tabular}
%\caption{The two TAC calibrations. TAC Calibration 1 is the calibration from the beginning of the experiment, just before data collection began. TAC Calibration 2 is from just after data collection ended.}
%\label{fig:calibration}
%\end{center}
%\end{figure}




\begin{thebibliography}{9}

\bibitem{lamport94}
  Leslie Lamport,
  \textit{\LaTeX: a document preparation system},
  Addison Wesley, Massachusetts,
  2nd edition,
  1994.

\bibitem{lmfit}
Newville, Matthew et al.. (2014). LMFIT: Non-Linear Least-Square Minimization and Curve-Fitting for Python. Zenodo. 10.5281/zenodo.11813



\end{thebibliography}







\end{multicols}

\end{document}

